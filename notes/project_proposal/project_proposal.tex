
% ======================================================================
% starting package maintenance...
% installation directory: C:\Users\damie\AppData\Local\Programs\MiKTeX
% package repository: https://ctan.math.illinois.edu/systems/win32/miktex/tm/packages/
% package repository digest: d8b7508a09a0813d998f77c3ce88406c
% going to download 74297 bytes
% going to install 42 file(s) (1 package(s))
% downloading https://ctan.math.illinois.edu/systems/win32/miktex/tm/packages/latexindent.tar.lzma...
% 0.07 MB, 1.03 Mbit/s
% extracting files from latexindent.tar.lzma...
% ======================================================================
\documentclass[12pt]{article}

% PACKAGES
\usepackage[margin = 0.6in]{geometry}
\usepackage{amsfonts}
\usepackage{amsmath}
\usepackage{amssymb}
\usepackage{multicol}
\usepackage{graphicx}
\usepackage{float}
\usepackage{xcolor}
\usepackage{amsthm}
\usepackage{dsfont}
\usepackage{hyperref}
\usepackage{setspace}
\usepackage{algorithm}
\usepackage{algpseudocode}
\usepackage{subcaption}
\newtheorem{theorem}{Theorem}
\newtheorem{problem}{Problem}
\setcounter{section}{-1}
% MACROS
% Set Theory
\def\N{\mathbb{N}}
\def\R{\mathbb{R}}
\def\C{\mathbb{C}}
\def\Z{\mathbb{Z}}
%\def\^{\hat}
\def\-{\vec}
\def\d{\partial}
\def\!{\boldsymbol}
\def\X{\times}
%\def\-{\bar}
\def\bf{\textbf}
\def\l{\left}
\def\r{\right}
\def\~{\tilde}
\def\t{\text}
\date{August 2022}
% \doublespacing
\title{Project Proposals}
\author{Damien Beecroft}
\date{August 2022}
\begin{document}
\maketitle
\section{Preface}

Hello Panos and Amanda, in the sections that follow I detail a few project ideas that I have for my PhD thesis in Applied Mathematics at the University of Washington. I am excited to share my ideas and I thank you for the opportunity to pursue them under your guidance. These proposals are very preliminary and are certainly subject to revision. All feedback is appreciated!

\section{Combining Numerics and PINNs for Optimal Modelling of Differential Equations}
\subsection{Abstract}
How does one appropriately combine the vast wealth of theory from numerical analysis and dynamical systems with physics informed neural networks to most efficiently and accurately solve differential equations? This is a complex problem whose answer depends heavily upon the equation, the available data, and a multitude of other factors. Wewill explore this trade-off to provide guidelines for how to incorporate data and theory effectively for a range of different scenarios through a variety of test problems with wildly different properties: the heat equation, the Helmholtz equation, Burger's equation, and Navier-Stokes equation.

\subsection{Introduction}
In this paper we study the efficacy of numerics assisted PINNs. In this work a numerics assisted PINN refers to 
an algorithm where a numerical method that serves as the low fidelity source to a multifidelity physics informed neural network (MFPINN) \cite{mfpinns}.
This framework is a natural way to incorporate numerical methods and PINNs. Furthermore, other methods (such as training a PINN to 
learn the residual of a numerical solution) can be interpreted as a sub-case of training a physics assisted PINN.
There are three broad metrics for success in the trials that we perform in this paper: accuracy, speed, and generalizability.
With these three variables we will construct the Pareto front of numerics assister PINNs as one shifts the computational burden between
the numerical method and the MFPINN. This will give the reader a holistic view of the trade-offs between numerics and machine learning
in physics applications.

\subsection{Body}
\subsubsection*{Accuracy and Speed}

From the studies of Kochkov et al. \cite{fluidml} we know that there exist scenarios in which machine learning
indeed extends the Pareto front of PDE solvers. Kochkov et al. \cite{fluidml} examines the usage of machine learning 
to improve finite volume method approximations of fluids with high accuracy solution data. 
They report that their method is roughly eighty times faster than the classical finite volume method.
Furthermore, their model is trained on local information and can therefore be scaled to be used for different initial and boundary conditions.
We will reproduce these results within these studies and subsequently extend upon them to see how the performance changes as one has less solution data
and must rely more and more on the physical constraints.

Incorporating a numerical method into the PINN framework will help overcome convergence issues that hinder the efficacy of PINNs.
Numerical methods do not have a spectral bias that makes it difficult to converge to oscillatory solutions \cite{bias}. Furthermore,
numerical algorithms do not converge to unstable fixed points as PINNs often do \cite{fixedpts}. There are many ways that numerics can 
correct pathological behavior in PINNs. It remains to be seen whether the converse also occurs. Can PINNs correct for the
biases of numerical methods? We will use simulations of finite volume methods on Burger's equation to tackle this problem. Certain finite
volume methods introduce artificial viscosity into the solution due to approximation error. We will examine whether
PINNs can detect that this phenomena is non-physical and subsequently correct for it.

The experiments will be performed as follows. For each differential equation we will choose a high accuracy numerical method and use it
as the true solution. Then, we will vary the amount of computational resouces of the numerical method and the MFPINN. For instance, say
that we have a set of grid sizes $G$ and a set of MFPINN sizes $N$. We will run numerics assisted PINNs on every combination of grid size in $G$ 
and network size in $N$. In otherwords, we will sample our numerics assisted PINN architecture from $G \times N$. We will run each model until it
has achieved the accuracy of the ``true solution" is reached or the method has hit a preset training time. These studies will be repeated for each
problem with a varying amount of solutions data. We will also perform extensive studies on problems with irregular domains since this 
has proven to be a major weakness of numerical methods. With all this information we hope to get a holistic view of when PINNs can improve upon classical
numerical schemes.

\subsubsection*{Generalizability}

Another important aspect of our tests is generalizability. It may be worth training a model for a long time given that it can solve a 
range of problems without retraining. We will take the numerics assisted PINNs trained in the last section
and see how well they perform on the same differential equations with different initial and boundary conditions. We will plot how the accuracy of
the numerics assisted PINNs changes depending on how different the initial and boundary conditions are from any of the problems in the training
data set. We expect differential equations with predominantly local dynamics (Burger's equation and Navier Stokes) to have an easier time generalizing than 
differential equations whose local dynamics are strongly coupled to the global state (the Helmholtz and heat equations).

\subsection{Expected Takeaways and ``Who Cares?''}

The field of physics informed machine learning has serious problems with reproducibility and benchmarking. 
Quality of PINN results are often dependent on hyperparameters that are tuned through an arduous cycle of trial and error.
This hyperparameter tuning is fine and good for showing how far one can push an algorithm. However, the need for tuning 
does not bode well for the robustness of PINNs. In these studies we intend to factor in the time required for hyperparameter tuning
into the comparison so that we may ascertain the strength of the method.

Adding numerics to the PINN training process undoubtedly makes convergence easier. If a PINN is incapable of improving
on a numerical method in this simpler scenario it is a sign that machine learning may not an effective tool for this particular
problem. Hopefully we will be able to extract patterns from these experiments and come up with guiding principles that can help
engineers make informed decisions on how to effectively combine the vast wealth of theory from numerical analysis 
and dynamical systems with physics informed neural networks.

\section{Analysis of Mini-Batch Gradient Descent for Physics Informed Neural Nets}

\subsection{Abstract}

The convergence behavior of PINNs is not well understood. Furthermore, most of the guidance we do have on hyperparameter selection comes from experimental data, not theoretical analysis. In this paper we analyze the PINN training process through the lens of minibatching stochastic gradient descent (MiniSGD) to produce theoretical principles that can aide one in choosing hyperparameters that lead to accelerated training.

\subsection{Introduction}

Suppose that we are attempting to solve the following evolution equation on some open solution domain $\Omega$ with boundary $\partial \Omega$ for time $t \in (0,t_{\text{f}}] = T$. $\Omega, \partial \Omega \subset \mathbb{R}^d$.

\begin{align} \label{eq:dq}
	u^*(t,x)_t + N[u^*(t,x)] = 0& \quad \text{for} \quad x \in \Omega, \: t \in T \notag\\
	B[u^*(t,x)] = 0& \quad \text{for} \quad x \in \partial \Omega , \: t \in T\\
	u^*(0,x) - u_0(x) = 0& \quad \text{for} \quad x \in \Omega \cup \partial \Omega \notag
\end{align} 

\noindent In Equation \ref{eq:dq}, $N$ and $B$ are differential operators and $u_0(x)$ is the initial state of the differential equation. We assume throughout this report that this differential equation is well posed with $u^*: \mathbb{R}^d \to \mathbb{R}^p$ as the true solution. Henceforth, we refer to $\Omega \cup \partial \Omega$ as $\overline{\Omega}$.

We want to train a neural network to solve Equation \ref{eq:dq}. We call this neural network $u(t,x;\theta)$. $\theta$ denotes the parameters of the neural net. We often drop the inputs or parameters of $u(t,x;\theta)$ for simplicity of notation, opting to call it $u(t,x)$ or $u$ instead. We would like this network to achieve a low loss in the following sense.\footnote{For now we ignore the regularization of the network through parameter magnitude penalization and the incorporation of solution data into the model. We will probably revisit the problem with these considerations later on.}
\begin{equation}
\mathcal{L_{\text{sup}}} = \sup_{(t,x) \in T \times\Omega} \| u_t + N[u] \|_{\infty} + \sup_{(t,x) \in T \times \partial \Omega} \| B[u] \|_{\infty} + \sup_{(t,x) \in \{0\} \times \overline{\Omega}} \| u - u_0 \|_{\infty}
\end{equation}

\noindent Unfortunately, the above loss is not feasible to optimize with gradient descent since it is not differentiable. We therefore turn our attention to a surrogate loss.

\begin{equation} \label{eq:Lint}
    \mathcal{L_{\t{int}}} = \int_{T} \int_{\Omega} \| u_t + N[u] \|_2^2 \t dx \t dt + \int_{T} \int_{\partial \Omega} \|B[u] \|_2^2 \t dx \t dt + \int_{\overline{\Omega}} \|u(0,x) - u_0(x) \|_2^2 \t dx
\end{equation} 

\noindent This loss function is differentiable.\footnote{I believe that we can bound $\mathcal{L}_{\t{sup}}$ in terms of $\mathcal{L}_{\t{int}}$ assuming that our objective function is Lipschitz. If we can do this we can achieve a type of uniform convergence of the solution which would be very nice.} However, these integrals cannot be computed analytically. In practice researchers sample collocation points separately from the sets $T \times \Omega$, $T \times \partial \Omega$, and $\{0\} \times \overline{\Omega}$. We refer to these as the solution, boundary, and initial domains respectively. These minibatches of collocation points are used to update the neural network parameters, $\theta$. For now, let us assume $(t,x)$ are sampled uniformly at random from each of the domains. The process we have just described culminates in algorithm \ref{alg:pinn}. 

\begin{algorithm}
\caption{Physics Informed Neural Network Training Process}\label{alg:pinn}
% \hspace*{\algorithmicindent} 
\begin{algorithmic}
    \State \textbf{Input:} Neural network: $u(t,x;\theta_0)$, Number of training iterations: J, Learning rate: $\epsilon$
    \For{j in 1:J}
    \State Sample $\{(t^s_j,x^s_j)\}_{j=1}^{N_s}$ where $(t^s_k,x^s_k) \thicksim \t{Unif}(T \times \Omega)$ \Comment{Sample solution domain points}
    \State Sample $\{(t^b_j,x^b_j)\}_{j=1}^{N_b}$ where $(t^b_k,x^b_k) \thicksim \t{Unif}(T \times \partial \Omega)$ \Comment{Sample boundary domain points}
    \State Sample $\{(t^i_j,x^i_j)\}_{j=1}^{N_i}$ where $(t^i_k,x^i_k) \thicksim \t{Unif}(\{0\} \times \overline{\Omega})$ \Comment{Sample initial domain points}
    \State $L_s \gets \sum_{j=1}^{N_s} \| u(t^s_j,x^s_j)_t + N[u(t^s_j,x^s_j)] \|_2^2$ \Comment{Solution loss}
    \State $L_b \gets \sum_{j=1}^{N_b} \| B[u(t^b_j,x^b_j)] \|_2^2$ \Comment{Boundary loss}
    \State $L_i \gets \sum_{j=1}^{N_i} \| u(t^i_j,x^i_j) - u_0(x^i_j) \|_2^2$ \Comment{Initial condition loss}
    \State $L \gets L_b + L_s + L_i$
    \State $\theta_j \gets \theta_{j-1} - \epsilon \nabla_{\theta_{j-1}} L$
    \EndFor
    \State \textbf{Return:} $u(t,x;\theta_J)$
\end{algorithmic}
\end{algorithm}

In this paper we study when we can expect Algorithm \ref{alg:pinn} to converge in the sense of $\mathcal{L_{\t{int}}}$ and $\mathcal{L_{\text{sup}}}$. Algorithm \ref{alg:pinn} is optimizing a highly non-linear function which makes analysis of the entire training process intractable. However, if we restrict our analysis to parts of the loss landscape that are locally convex--where we will find minima--we can begin to understand the convergence behavior.

\subsection{Minibatching Stochastic Gradient Descent}

Before we dive into the analysis of PINN convergence, we first give a synopsis of relevant results in minibatching stochastic gradient descent. In Garrigos et al. \cite{gd} they analyze the following scenario.\footnote{I strongly encourage reading through section 6 of Garrigos et al. \cite{gd} if you want to understand what follows.}

\begin{problem} \label{prob:sum}
    (Finite sum of functions) We want to minimize $f: \mathbb{R}^d \to \mathbb{R}$
    where
    \begin{equation} \label{eq:f}
        f(\theta) \equiv \frac{1}{M} \sum_{i=1}^{M} f_i(\theta)
    \end{equation}
    and $f_i: \mathbb{R}^d \to \mathbb{R}$. Furthermore, we assume that each $f_i$ is convex and $L_i$-smooth. We define $L_{max} \equiv \max_{i \in \mathbb{N}_{\leq M}} L_i$ where $\mathbb{N}_{\leq M} \equiv \{1,2,\hdots, M \}$.
\end{problem}

The minimization of Equation \ref{eq:Lint} does not exactly fit into this framework. However, we will come back to this issue later. Suppose that $M$ is large and instead of performing gradient descent on all of $f$, we would prefer to perform minibatching stochastic gradient descent (MiniSGD) to find the optimum. Let $B_t \subset \mathbb{N}_{\leq M}$, $|B_t| = m < M$. Every iteration of MiniSGD the indices, $B_t$, are resampled. The subscript $t$ denotes the MiniSGD iteration. The MiniSGD step is defined as follows.

\begin{align}
    \nabla f_{B_t}(\theta_t) &\equiv \frac{1}{m} \sum_{i \in B_t} \nabla f_i(\theta_t)\\
    \theta_{t+1} &= x_t - \gamma_t \nabla f_{B_t}(\theta_t)
\end{align}

Garrigos et al. \cite{gd} provide the following convergence bound for the application of MiniSGD to Problem \ref{prob:sum}.

\begin{theorem} \label{thm:sgd}
    Let $f(\theta_*) = \inf f$. Consider a sequence $(\theta_t)_{t \in \mathbb{N}}$ generated by the MiniSGD algorithm with stepsizes $0 < \gamma_t < \frac{1}{2 \tilde{L}}$. It follows that

    \begin{equation}
        \mathbb{E}[f(\overline{\theta}_t) - \inf f] \leq \frac{\|\theta_0 - \theta_*\|^2}{2 \sum_{k=0}^{t-1} \gamma_k(1-2\gamma_k \tilde{L})} + \frac{\sigma_b^* \sum_{k=0}^{t-1} \gamma_k^2}{\sum_{k=0}^{t-1} \gamma_k (1 - 2 \gamma_k \tilde{L})} 
    \end{equation}

    where $\overline{\theta}_t \equiv \sum_{k=0}^{t-1} p_{t,k} \theta_k$, with $p_{t,k} \equiv \frac{\gamma_k (1 - 2 \gamma_k \tilde{L})}{\sum_{i=0}^{t-1} \gamma_i (1 - 2 \gamma_i \tilde{L})}$. $\tilde{L}$ is the expected smoothness constant of $f_B$ and $\sigma_b^*$ is the variance of the loss gradient at the optimum point, $\theta_*$.
\end{theorem}

We don't want to dive into too much detail about the above theorem. A detailed explanation can be found in section 6 of Garrigos et al. \cite{gd}. The important part to take away from Theorem \ref{thm:sgd} is that the convergence rate is highly dependent on the smoothness of the objective function, the noisiness of the gradient, and the step sizes. 

As mentioned before, this framework does not directly apply to the optimization of PINNs. The difference between MiniSGD as described above and the PINN learning process is that there are a countable number of potential batches in the normal MiniSGD algorithm whereas during the PINN training process there are an uncountably infinite number of collocation points (and hence batches) that can be selected at each step. We want to derive an analog of Theorem \ref{thm:sgd} that applies to Equation \ref{eq:Lint} so that we may analyze the relationship between the density of collocation points used in each batch of Algorithm \ref{alg:pinn} and the rate of convergence. Once this new convergence bound is derived we plan to use PDE theory to find bounds on the smoothness and gradient noise. With this theory we will derive stipulations on the learning rate and number of collocation points that should be used for minimizing Equation \ref{eq:Lint} under the assumption that $\theta$ is convex. Our assumptions of convexity are not true globally, but they may hold near $\theta_*$. This analysis may help us understand how to improve the accuracy of PINNs given that we are able to get close enough to $\theta_*$. We will also do experiments to see whether the theory that comes from the MiniSGD analysis helps convergence.

% However, if we consider an objective function $f$ that is composed of the sum of an uncountably infinite number of functions as opposed to a countable number we can begin to study equation \ref{eq:Lint} in a similar manner that Garrigos et al. \cite{gd} study equation \ref{eq:f}. More precisely, we are interested in the following set up.

% \begin{problem} \label{prob:sum2}
%     (Finite sum of functions) We want to minimize $f: \mathbb{R}^d \to \mathbb{R}$
%     where
%     \begin{equation} \label{eq:f2}
%         g(\theta) \equiv \int_{\Gamma_i} \tilde{g}(u; \theta) \emph{d}u
%     \end{equation}
%     and $\tilde{g}: \mathbb{R}^l \times \mathbb{R}^d \to \mathbb{R}$. Furthermore, we assume that each $\tilde{g}$ is convex in $\theta$ and $L(u)$-smooth. We define $L_{max} \equiv \sup_{u \in \Gamma} L(u)$.
% \end{problem}

\bibliographystyle{plain}
\bibliography{refs}
\end{document}


% Before I dive into the proposals I thought it beneficial to give a brief synopsis of my views on the field of physics informed machine learning and the role I envision myself playing in this field.

% During the course of my research, I experienced first-hand the training difficulties of PINNs. I often felt frustrated for the lack of theory 


% Many of the hyperparameters used in training are selected ad hoc, despite having significant influence on the success of training. How does one appropriately choose collocation points 

% The convergence rates of numerical methods are heavily dependent on the density of collocation points in the solution domain. Physics informed neural networks (PINNs) attempt to solve the same problems, however there are no theoretical results that provide guidance as to the quantity and density of collocation points in each batch to achieve convergence.

% In this paper we try to give a theoretical basis for the 

% \newpage
% i \in \{1,2,\hdots, n \}
% \subsection*{Project Two}
% After the PINN experiments are done I want to repeat the studies from project one with multifidelity finite basis DeepONets. 
% DeepONets do not work well alone. I wonder whether the aide of numerical analysis can make DeepONets viable. This is the 
% very important experiment. Having to retrain a network for every set of initial conditions is unreasonable in a wide range of applications.
% \subsection*{Project Three}
% I think it would be interesting to look at the loss landscape of the MFFBPINNs with numerical solvers. We know that at 
% the beginning of training you are within a certain error $\epsilon$ of the true solution. This means that evaluating a linear
% or quadratic approximation of the neural network at initialization would likely give one a good approximation of the loss landscape at the minima. Furthermore, there is something that has struck me as a bit strange about the PINN training process.
% In numerical methods the density of grid points is paramount to successful convergence. However, density of collocation points in batches for PINNs is not--at least in my experience--treated with the same importance. When one creates a batch of collocation points 
% I believe they should be sampled at or above the Nyquist frequency of the highest frequency mode in the solution. I am curious to see 
% whether my intuition can be tested by analyzing the loss landscape. It may also be interesting to analyze the convergence of MFFBPINNs with
% numerical solvers through a combination of numerical solvers and the neural tangent kernel.

%During the course of this internship, we developed two main ways to implement this algorithm. Each one is centered around how batching is done. In the first implementation, a batch containing points in the entire domain $\Omega$ is sent to the MFFBPINN. In this construction we use a tree of neural networks. The zeroth level creates its solution prediction. The networks on the first level (which are all children of the single fidelity network on the zeroth level) then look at the points in the original batch and the low fidelity solution from the zeroth layer and make their own approximation for the points that are within their subdomains. The child of each network in the second level then take in the batch points and approximations of their parents and create new approximations on the points in their subdomains and so on and so forth. A diagram illustrating this process is given in figure \ref{fig:batch1}. This method scales well with higher dimension and is rather flexible with respect to the shape of the subdomains.

%In the second implementation of the MFFBPINN, distinct batches of residual points are created for each intersection of the subdomains on the highest level. These batch points are then fed down to the neural networks on the lower levels to get the low fidelity approximations. This method is illustrated in Figure
% None of the work on physics informed neural networks (PINNs) that I have seen leverages the vast wealth of 
% existing numerical methods.

% How does one appropriately combine the vast wealth of theory from numerical analysis 
% and dynamical systems with machine learning and data science to most efficiently and 
% accurately solve differential equations? This is a complex problem whose answer depends 
% heavily upon the equation, the available data, and a multitude of other factors. I will
% explore this trade-off to provide guidelines for how to incorporate data and theory effectively 
% for a range of different scenarios. In this paper I explore a variety of test problems, each
% with wildly different theoretical properties: the heat equation, the Helmholtz equation, Burger's equation, 
% and Navier-Stokes equation. This group of diverse equations will help us achieve a holistic perspective
% on how and when to leverage data science. 